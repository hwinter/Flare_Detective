%this is short summary of useful formulae for electron <-> photon
%spectra conversion

\documentclass[11pt]{article}
\usepackage{a4,}
%\documentclass{aa}
%\usepackage{txfonts}
%\usepackage{graphicx}

%\onecolumn

\begin{document}
This is a short summary of some useful formul\ae\/ for the conversion
between photon and electron spectra.
Let's define:
\begin{itemize}
\item
$F(E)$ is the hard X-ray flux at energy $E$ observed at 1 AU. Its units are
photons $\mathrm{s}^{-1}$ $\mathrm{cm}^{-2}$ $\mathrm{keV}^{-1}$. 
\item
$\Phi (\epsilon)$ is the total electron flux at the sun. Its units are
electrons $\mathrm{s}^{-1}$ $\mathrm{keV}^{-1}$. 
\end{itemize}
Using the nonrelativistic Bethe-Heitler bremsstrahlung cross section, and
assuming an electron spectrum $\Phi (\epsilon)=A\epsilon^{-\delta}$, the 
resulting thick target photon spectrum is given by (Lang \cite{lang98})
\begin{equation}
F(E)=A \frac{\kappa_{\mathrm{BH}} \overline{Z^2}}{4\pi R^2 C}
\frac{B(\delta-2,1/2)}{(\delta-1)(\delta-2)} E^{1-\delta}
\, ,
\end{equation} 
where
\begin{eqnarray*}
R                    & = & \mathrm{1 AU}=1.496\cdot 10^{13} \mathrm{cm}\\
C                    & = & 2\pi e^4 \ln\Lambda = \ln\Lambda \cdot 1.303 \cdot 10^{-19} \mathrm{cm^2 keV^2} \\
\kappa_{\mathrm{BH}} & = & \frac{8}{3} \alpha r_e^2 m_e c^2 = 7.896 \cdot 10^{-25} \mathrm{cm^2 keV} \\
\overline{Z^2}       &   & \mbox{is the average atomic number, about 1.4 for solar conditions} \\
\ln\Lambda           &   & \mbox{is the Coulomb logarithm, about 20 for solar conditions} \\
B(x,y)               & = & \frac{\Gamma(x)\Gamma(y)}{\Gamma(x+y)} \quad \mbox{is the Beta function}
\end{eqnarray*} 

Using a slightly different notation:
\begin{eqnarray*}
F(E) & = & F_0 \left(\frac{E}{E_0}\right)^{-\gamma} \\
\Phi(\epsilon) & = & \Phi _0 \left(\frac{\epsilon}{\epsilon_0}\right)^{-\delta}
\end{eqnarray*}
then
\begin{eqnarray}
\delta & = & \gamma + 1\\
\Phi_0 & = & K\,{F_0}\,{E_0}^\gamma\,{\epsilon_0}^{-\gamma-1}
             \frac{\gamma(\gamma-1)}{B(\gamma-1,1/2)}\\
F_0    & = & \frac{1}{K}\,{\Phi_0}\,{E_0}^{1-\delta}\,{\epsilon_0}^{\delta}
             \frac{B(\delta-2,1/2)}{(\delta-1)(\delta-2)}
\end{eqnarray}
with 
\begin{eqnarray*}
K & = & \frac{4\pi R^2 C}{\kappa_{\mathrm{BH}}\overline{Z^2}} =
4.640 \cdot 10^{32} \,\mathrm{cm}^2\,\mathrm{keV} \,\frac{\ln\Lambda}{\overline{Z^2}}=
6.44  \cdot 10^{33} \,\mathrm{cm}^2\,\mathrm{keV}
\end{eqnarray*}


\begin{thebibliography}{}
\bibitem[1998]{lang98}
Lang, Kenneth R., Astrophysical Formul\ae , Springer, 1998
\end{thebibliography}



\end{document}